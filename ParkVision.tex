% ----------
% A LaTeX template for course project reports
% 
% This template is modified from "Tech Report ala MIT AI Lab (1981)"
% 
% ----------
\documentclass[12pt, letterpaper, twoside]{article}
\usepackage{geometry}
\usepackage[utf8]{inputenc}
\usepackage[english]{babel}
\usepackage[runin]{abstract}
\usepackage{titling}
\usepackage{booktabs}
\usepackage{fancyhdr}
\usepackage{helvet}
\usepackage{csquotes}
\usepackage{graphicx}
\usepackage{blindtext}
\usepackage{parskip}
\usepackage{etoolbox}


\input{preamble.tex}

% ----------
% Variables
% ----------

\title{\textbf{Computer Vision Course Project:\\Park Vision: An Innovative Computer Parking Solution}} % Full title of your tech report
\runningtitle{ParkVision} % Short title
\author{Aaron James \and Jacob Rempel} % Full list of authors
\runningauthor{} % Short list of authors
\affiliation{Ontario Tech University} % Affiliation e.g. University or Company
\department{Faculty of Science, Computer Science} % Department or Office
\memoid{Project Group: 19} % Project group ID that were shared with the class earlier.
\theyear{2025} % year of the tech report
\mydate{April, 6, 2025} %the date


% ----------
% actual document
% ----------
\begin{document}
\maketitle

\begin{abstract}
    \noindent
    
Efficient parking management is a growing challenge in many urban environments, leading to issues such as wasted time, driver frustration and increased fuel consumption. By design ParkVision is a computer vision project that rapidly finds vacant parking spots from an aerial view. By leveraging image processing, deep learning models, and real-time video analysis, this system determines the occupancy of spots by applying computer vision techniques to still frames extracted from a live video feed. The scope of this project is parking lots with straight and uniform spots. Time permitting, we may extend the project to identify irregular or angular parking spots. The goal of this system will be to provide real-time insight for drivers to improve parking traffic flow and management.



    % Uncomment the following to add keywords as needed
    % \keywords{Keyword1, Keyword2, Keyword3}
\end{abstract}

\vspace{2.5cm}

% Uncomment the following to add thanks.
% {\footnotesize
%     \noindent
%     Special thanks to \textbf{Person 1} and \textbf{Affiliation A} for financial support for this project.
% }

\thispagestyle{firstpage}

\pagebreak

% ----------
% End of first page
% ----------

\newgeometry{} % Redefine geometries (normal margins)

\section{Introduction}
\label{sec:intro}

\blindmathpaper % Delete this line.  It just creates some text for display purposes.

\section{Conclusions}
\label{sec:conc}

Tech Report writing is an art.

% Uncomment following to add an acknowledgement section
% \section*{Acknowledgements}

% Thanks again to \textbf{Person 1} and \textbf{Affiliation A} for their financial support.

% ----------
% Bibliography
% ----------

% Uncomment the following and add your references into biblio.bib file
% \bibliography{./biblio.bib}
% \bibliographystyle{abbrv}

\appendix

\section{An appendix}

\blindtext % Delete this line.  It just creates some text for display purposes.

\section{Another appendix}

\subsection*{Subsectioning in appendix}

Some more text, and a list:

\blindenumerate % Delete this line.  It just creates some text for display purposes.

\end{document}

% ----------
